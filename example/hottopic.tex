\documentclass{../khw}

\usepackage{lipsum}

\title{Student Loan}{Debt Forgiveness}
\subtitle{Hot Topic}
\author{
    Kian Kasad\\[4pt]
    Shawn Reznikov\\[4pt]
    Ten Chanyontpatanakul
}
\course{AP Government}
\instructor{B. Andrews}

% Abbreviations
\newcommand{\ed}{Department of Education}
\newcommand{\used}{U.S. \ed}

\begin{document}
\maketitle

\section{Types of Federal Student Aid}

Aid comes in two forms: grants and loans. Grants are like gifts, they do not
need to be repaid. Loans must be paid back in full over time, plus an additional
amount in interest.

There are two common types of federal student loans: Direct Loans and Federal
Family Education Loans (FFELs). Direct Loans are loans issued directly by the
\used (ED) using funds from the Federal Reserve. FFELs are loans issued by
private lenders and backed by the ED.

\section{Debt Forgiveness}

There are several types of debt forgiveness. Most of these programs were created
during the COVID-19 pandemic to lessen financial struggles during an already
stressful time.

\subsection{Public Service Loan Forgiveness}

This program was not created in response to COVID-19. It was
established in 2007.

\blockquote{
    Public Service Loan Forgiveness (PSLF) is a federal loan forgiveness program
    established as part of the College Cost Reduction and Access Act of 2007.
    Under PSLF, eligible borrowers who work full-time for nonprofit
    organizations or federal, state, local or tribal governments can get loan
    forgiveness after making 120 qualifying monthly payments.
}{Forbes}%
{https://www.forbes.com/advisor/student-loans/public-service-loan-forgiveness-program/}

This means after making payments for 10 years, any outstanding loan balance will
be forgiven for qualifying borrowers. This is a pre-existing program and is not
a part of the Biden Administration's new plan.


\subsection{Payment Pause}

\blockquote{
    A payment pause is currently in effect, meaning borrowers do not have to make
    any payments on their loans. No additional interest is incurred either. If your
    loans are eligible, we automatically paused your loan payments and set your
    interest rate to 0\% starting March 13, 2020. This payment pause, also known as
    the administrative forbearance, will end Dec. 31, 2022.
}{\used}%
{https://studentaid.gov/announcements-events/covid-19/payment-pause-zero-interest}

Payments will resume on January 1, 2023.

\subsection{One-Time Relief}

\blockquote{
    The U.S. Department of Education will provide up to \$20,000 in debt relief to
    Pell Grant recipients with loans held by the Department of Education and up to
    \$10,000 in debt relief to non-Pell Grant recipients. Borrowers are eligible for
    this relief if their individual income is less than \$125,000 or \$250,000 for
    households.
}{\used}%
{https://studentaid.gov/debt-relief-announcement/}

Applications for this one-time relief are expected to open in October 2022 and
are open through December 31, 2023. Qualifying recipients of at least one Pell
Grant will automatically have \$20,000 forgiven. Those who did not receive a
Pell Grant during college will have \$10,000 forgiven.

FFEL program loans which are not held by the \used are not eligible for relief,
as they're issued by private lenders and not the U.S. government.

Borrowers who voluntarily made payments during the payment pause will have their
outstanding debt forgiven, as well as any payments that brought their balance
below the maximum amount they qualify for. For example, someone with \$12,000 of
debt paid off \$3,000 during the pause, bringing their balance to \$9,000. The
remaining \$9,000 will be forgiven and \$1,000 of their payments will be
refunded.

\subsubsection{Legal Justification}

\fntext{lawsuits}{\url{
    https://www.msn.com/en-us/money/careersandeducation/bidens-student-loan-forgiveness-could-be-delayed-next-week-as-lawsuits-continue-to-mount-heres-where-the-cases-stand/ar-AA12Ldf1\#image=AA12KXdV|4
}}

According to Business Insider author Ayelet Sheffey,\fn{lawsuits}
``\textit{%
    The White House has maintained it has the authority to enact this one-time broad
    relief under the HEROES Act of 2003, which gives the Education Secretary the
    authority to modify or waive student-loan balances in connection with a national
    emergency, like COVID-19.%
}''

\section{Current Debt Levels}

\fntext{distdata}{\url{
    https://studentaid.gov/sites/default/files/fsawg/datacenter/library/Portfolio-by-Debt-Size.xls
}}

\fntext{totaldebt}{\url{
    https://www.federalreserve.gov/datadownload/Download.aspx?rel=g19&series=49035952ad4d97a13e2eef63bb7e342c&filetype=csv&label=include&layout=seriescolumn&from=01/01/2020&to=12/31/2022
}}

As of June 2022, the total amount of federal student loan debt is
\$1,745,369,310 (\$1.7 trillion).\fn{totaldebt} 26\% of borrowers have less than \$10,000 in
student loan debt. 47\% of borrowers have less than \$20,000 of debt.\fn{distdata} This means
that around 25-50\% of all borrowers would be completely free of their student
loan debt under Biden's loan forgiveness plan.

\begin{figure}[h]
    \centering%
    \framedgraphics{loan_size_distribution}%
    \caption{Distribution of federal student loans by loan amount.}
    \label{fig:loandist}%
\end{figure}

The typical undergraduate student with loans now graduates with nearly \$25,000
in debt.

\section{Opposition to Biden's Plan}

The Biden Administration's plan is not seen as a good thing by everyone.

\subsection{Legal Arguments}

Many conservatives claim the plan is “unfair, costly, and illegal.” Their
argument is that all taxpayers will have to shoulder the cost, but the benefits
are only accessible to a select group. Four lawsuits have been filed against the
Biden Administration's plan. One of them has already been struck down by the
suit's judge.\fnref{lawsuits}

On Monday, Oct 10, another lawsuit was filed against the debt
relief program, arguing that the Biden Administration violated federal procedure
by failing to seek public input on the plan. Their position is that the
Administration is seeking to pass the program before midterm elections in
November.\footnote{\url{
    https://www.pbs.org/newshour/education/new-lawsuit-from-small-business-group-seeks-to-block-biden-student-debt-relief-plan
}}

\subsection{Economic Arguments}

One argument for the plan is that it will provide an economic stimulus during a
recession. If people have less debt, their monthly payments will be lower,
allowing them to spend the money that they would've been spending to pay off
loans.

However, this is not a valid argument, as there is already a payment pause in
effect through the end of 2022, meaning debt forgiveness would provide no
immediate economic stimulus.

The relief plan may actually worsen the
recession, as releasing hundreds of billions of dollars into the economy will
cause inflation, as people now have more expendable income to spend on the same
amount of products.

Another economic argument against the plan is that it will result in a loss of
revenue for states and for private student loan companies like MOHELA. MOHELA
services FFEL loans, which are not eligible for debt relief unless they're
consolidated into the Direct Loan program. This encouragement to consolidate
will result in a loss of revenue for companies like MOHELA.\fnref{lawsuits}

\subsection{Moral Arguments}

Some argue that forgiving debt creates certain incentives which are overall
harmful. Even though some debt may be forgiven now, new loans are still being
issued. Student debt levels will continue to rise and will reach their current
levels in just a few years. If another cancellation happens in the future, there
is an incentive to borrow more money with the expectation that at least some of
it will be forgiven. This incentivises colleges to raise tuition even higher and
students to borrow more money.\footnote{\url{
    https://www.forbes.com/sites/prestoncooper2/2020/11/17/the-case-against-student-loan-forgiveness/
}}

\section{More Resources}

YouTube Videos:
\begin{itemize}
    \item \href{https://www.youtube.com/watch?v=--ixiTypG8g}
        {How Biden's student loan forgiveness program will work}
    \item \href{https://www.youtube.com/watch?v=-uhxsF1mhv8}
        {Explained: How to Prepare for Student Loan Debt Forgiveness}
    \item \href{https://www.youtube.com/watch?v=BtCwtuc63Hw}
        {Explaining Biden's student loan forgiveness policy}
        (possibly good for inclusion in slideshow)
\end{itemize}

\section{Controversy Summary}

\subsection{Pro Arguments}
\begin{itemize}
    \item Helps resolve student debts
    \item Stimulates economy during the current recession
\end{itemize}

\subsection{Con Arguments}
\begin{itemize}
    \item Unfair distribution of money
    \item Unfair to those who already paid their student loans (before payment pause)
    \item Places burden on taxpayers
    \item Creates precedent for student loan debt cancellation in the future
    \item Adds to the government's debt, likely increasing inflation
    \item Plan is rushed and secretive, as if trying to rally support before midterm elections
\end{itemize}

\section{Burning Questions}

\begin{itemize}
    \item Is student debt cancellation fair to those who have already paid off their student loans?
    \item Will the relief program help or hurt the economy?
    \item Is the Biden Administration's plan the best way to handle the current student debt crisis?
    \nestitem Follow up:   Is this the problem Biden should be focusing on now?

    \item How does the debt relief plan affect the cost of college moving forwards? Is it fair to those who don't take out loans?
    \nestitem Follow up:   Does this encourage or discourage the payment of loans, and is it a good thing?
\end{itemize}

\textbf{CAN ONLY HAVE 3-4.}

\section{Test}
This section tests single-word section headings.


\end{document}
